%%%%%%%%%%%%%%%%%%%%%%%%%%%%%%%%%%%%%%%%%
% Jefferson Ding's Writing Template
% Modified from Jefferson's IB Template
% Modified from OIST Doctoral Thesis
% LaTeX Template
% Version 0.1 (2021/11)
% Version 1.0 (2023/8)
%
% Author:
% Jefferson Ding
% Original author:
% Jeremie Gillet
%
%%%%%%%%%%%%%%%%%%%%%%%%%%%%%%%%%%%%%%%%%

%-------------------------------------------------------------------------------
%	REQUIRED PACKAGES AND  CONFIGURATIONS
%-------------------------------------------------------------------------------

\documentclass[final]{writing_template} 

% The documentclass ib_template includes the following packages: geometry, caption, xkeyval

\usepackage[english]{babel} % The document is in English
\usepackage[utf8]{inputenc} % UTF8 encoding
\usepackage[T1]{fontenc} % Font encoding

\usepackage{graphicx} % For including images
\graphicspath{{./img/}} % Specifies the directory where pictures are stored

\usepackage{setspace} % For using single or double spacing
\usepackage{longtable} % tables that can span several pages
\usepackage{pdfpages} % To include a pdf files of your published papers as an appendix
\usepackage{fancyhdr} % For the headers
\usepackage{hyperref} % Adds clickable links at references
\setcounter{tocdepth}{4}
\setcounter{secnumdepth}{4}
%----------------------------------------------------------------------------------------
%	ADD YOUR PACKAGES (be careful of package interaction)
%----------------------------------------------------------------------------------------

\usepackage{amsthm,amsmath,amssymb,amsfonts,bbm}% Math symbols
\usepackage{framed}
\usepackage{tikz}
\usepackage{pgfplots}
\pgfplotsset{compat=1.18}
\usepackage{float}
\usepackage{booktabs}
\usepackage{xcolor}% or package color
\usepackage{tocloft}
%----------------------------------------------------------------------------------------
%	ADD YOUR DEFINITIONS AND COMMANDS
%----------------------------------------------------------------------------------------
\newcommand{\quickwordcount}[1]{%
  \immediate\write18{texcount -1 -sum -merge -q #1.tex > #1-words.sum }%
  \input{#1-words.sum}%
}

\newcommand{\e}[1]{\times 10^{#1}}  % Powers of 10 notation

\makeatletter
\renewcommand*\env@matrix[1][*\c@MaxMatrixCols c]{%
  \hskip -\arraycolsep
  \let\@ifnextchar\new@ifnextchar
  \array{#1}}
\makeatother


%\newtheorem{theorem}{Theorem}

\newtheorem{theorem}{Theorem}[section]
\newtheorem{corollary}{Corollary}[theorem]
\newtheorem{lemma}[theorem]{Lemma}
%\newtheorem{proof}{Proof}[theorem]
\renewcommand\qedsymbol{$\blacksquare$}

\usepackage{xpatch}
\makeatletter
\AtBeginDocument{\xpatchcmd{\@thm}{\thm@headpunct{.}}{\thm@headpunct{}}{}{}}
\makeatother

%-------------------------------------------------------------------------------
%	TITLE PAGE
%-------------------------------------------------------------------------------

\begin{document}
\pagestyle{empty} % No page numbers

\puttitle{
	papertitle={Theorems, Lemmas \& Proofs},% Title of the paper
  %rq = {Calculus II Cheatsheet},
  school = {University of Pennsylvania},
  subject = {CIS 1600},
	name = {Jefferson Ding},
  submissiondate = {F23},
}



%-------------------------------------------------------------------------------
%	PREAMBLE PAGES (comment unnecessary pages)
%-------------------------------------------------------------------------------

\startpreamble









%-------------------------------------------------------------------------------
%	LIST OF CONTENTS/FIGURES/TABLES
%-------------------------------------------------------------------------------

\unnumberedsection{Contents}
\tableofcontents % Write out the Table of Contents
\singlespacing



%\unnumberedsection{Table of Equations}
%\singlespacing
%\listofmyequations
%\tableofcontents % Write out the Table of Contents



%-------------------------------------------------------------------------------
%	PAPER MAIN TEXT
%-------------------------------------------------------------------------------

\addtocontents{toc}{\vspace{1em}} % Add a gap in the Contents, for aesthetics

\numberedsection
% Review
%\input{src/section0} 
% Vectors, Lines and Planes
%\input{src/section1}
% Vector Calculus
%\input{src/section2}
% Matrices
%\input{src/section3}
% Coordinate Systems
%\input{src/section4}
% Determinants
%\input{src/section5}
 %Derivatives
%\input{src/section6}
%\input{src/week1}
%\input{src/week2}
%\input{src/week3}
%\input{src/week4}
\section{Logic \& Number Theory}
\subsection*{Logic}
\begin{theorem}
	$$
		p\rightarrow q \equiv \neg p \lor q \equiv \neg q \rightarrow \neg p
	$$
\end{theorem}
\begin{theorem}[Basis for Contradiction]
	$$
	p\equiv \neg p \rightarrow C \qquad p\rightarrow q \equiv p \land \neg q \rightarrow C
	$$
\end{theorem}

\subsection*{Number Theory}
\begin{theorem}
	If the sum of two integers is even, their difference is also even.
\end{theorem}

\begin{theorem}
If the product of $n$ integers is odd, then all of the integers are odd.
\end{theorem}

\begin{theorem}
	If $n$ is odd, then $n^2+n+1$ is odd.
\end{theorem}

\begin{theorem}
	For all $x\in \mathbb{Z}^+$, $x^3+1$ is composite
\end{theorem}

\begin{theorem}
	For all $x,y\in \mathbb{Z}$, if $x+y$ is even, then $x$ and $y$ are both odd or both even.
\end{theorem}

\begin{theorem}
	For all $x,y\in \mathbb{R}$, if $ab$ is irrational, then either $a$ or $b$ or both are irrational.
\end{theorem}

\begin{theorem}[Sum of Arithmetic Series]
	For all $n\in \mathbb{Z}^+$, $\sum_{i=1}^n i = \frac{n(n+1)}{2}$
\end{theorem}

\begin{theorem}[Sum of Geometric Series]
	For all $n\in \mathbb{Z}^+$, $\sum_{i=1}^n r^i = \frac{r^{n+1}-1}{r-1}$
\end{theorem}

\begin{theorem}[Sum of Infinite Geometric Series]
	For all $|r| < 1$, $\sum_{i=1}^\infty r^i = \frac{1}{1-r}$
\end{theorem}

\begin{theorem}[Unique Factorization Theorem]
	For all $n\in \mathbb{Z}^+$, $n$ can be written as a product of primes in a unique way.
\end{theorem}

\begin{theorem}
	There exists infinitely many primes.
\end{theorem}

\begin{theorem}[Binomial Theorem]
	For all $n\in \mathbb{Z}^+$, $(x+y)^n = \sum_{i=0}^n \binom{n}{i} x^{n-i}y^i$
\end{theorem}

\begin{theorem}[Generalized Pigeonhole Principle]
	For all $n\in \mathbb{Z}^+$, to place $n$ objects into $k$ boxes, at least one box must contain $\lceil \frac{n}{k} \rceil$ objects.
\end{theorem}

\begin{theorem}
	For a sequence of $n$ positive integers, there exists a consecutive subsequence whos sum is devisable by $n$
\end{theorem}

\section{Sets}
\begin{theorem}
	$$
	|P(A)| = 2^{|A|}
	$$
\end{theorem}

\begin{theorem}
	$\mathcal{P}(A) \cup \mathcal{P}(B) = \mathcal{P}(A \cup B)$
\end{theorem}

\begin{theorem}
	$(T\cap S)\cup (F\cap G) \subseteq (T\cup F)\cap (S\cup G)$
\end{theorem}

\begin{theorem}
	$A\times B = B\times A \leftrightarrow A=B$
\end{theorem}

\begin{theorem}[De'Morgan's Laws]
	\begin{equation*}
		\begin{split}
			A-(B\cup C) = (A-B)\cap (A-C)\\
			A-(B\cap C) = (A-B)\cup (A-C)
		\end{split}
	\end{equation*}
\end{theorem}

\begin{theorem}[Principal of Inclusion-Exclusion]
	$$
	|A_i\cup \hdots\cup A_n| = \sum_{i=1}^n |A_i| - \sum_{i<j} |A_i \cap A_j| + \sum_{i<j<k} |A_i \cap A_j \cap A_k| - \hdots + (-1)^{n+1} |A_1 \cap A_2 \cap \hdots \cap A_n|
	$$
\end{theorem}

\section{Counting \& Probability}
\begin{theorem}[Multiplicatoin Rule]
	For all $n\in \mathbb{Z}^+$, if a task consists of $n$ steps, where the $i$th step can be done in $n_i$ ways, then the task can be done in $\prod_{i=1}^n n_i$ ways.
\end{theorem}

\begin{theorem}[Permutation of Multisets]
	If there ar $n_1$ $T_1$ objects, $n_2$ $T_2$ objects, $\hdots$, $n_k$ $T_k$ objects, then there are
	$$
	\frac{(n_1 + n_2 + \hdots + n_k)!}{n_1!n_2!\hdots n_k!}
	$$
	ways to arrange them.
\end{theorem}

\begin{theorem}[r-Combinations with Repetition]
	For all $n,r\in \mathbb{Z}^+$, there are
	$$
	\binom{n+r-1}{r}
	$$
	ways to choose $r$ objects from $n$ types of objects with repetition.
\end{theorem}

\begin{theorem}[Pascals Formula]
	For all $n,r\in \mathbb{Z}^+$, $\binom{n}{r} = \binom{n-1}{r-1} + \binom{n-1}{r}$
\end{theorem}

\begin{theorem}
	$\sum_{i=0}^n \binom{n}{i} = 2^n$
\end{theorem}

\subsection*{Probability}
\begin{theorem}[Conditional Probability]
	For all events $A,B$,
	$$
	Pr[A|B] = \frac{Pr[A\cap B]}{Pr[B]}
	$$
\end{theorem}

\begin{corollary}
	For all events $A,B$,
	$$
	Pr[A\cap B] = Pr[A|B]Pr[B]
	$$
\end{corollary}

\begin{theorem}[Total Probability Theorem]
	For all events $A,B_1,\hdots,B_n$,
	$$
	Pr[A] = \sum_{i=1}^n Pr[A|B_i]Pr[B_i]
	$$
\end{theorem}

\begin{theorem}
	Event $A$ and $B$ are independent $\leftrightarrow$ $Pr[A\cap B] = Pr[A]Pr[B]$
\end{theorem}

\begin{theorem}[Expected Value]
	For all events $A_1,\hdots,A_n$,
	$$
	E[X] = \sum_{i=1}^n Pr[A_i]X_i
	$$
\end{theorem}

\begin{theorem}[Linearity of Expectation]
	For all events $A_1,\hdots,A_n$,
	$$
	E[X] = \sum_{i=1}^n E[X_i]
	$$
\end{theorem}

\begin{theorem}[Markov's Inequality]
	For all random variables $X$ and $a>0$,
	$$
	Pr[X\geq a] \leq \frac{E[X]}{a}
	$$
\end{theorem}

\begin{theorem}[Varience]]
	For all random variables $X$,
	$$
	Var[X] = E[X^2] - E[X]^2
	$$
\end{theorem}

\begin{theorem}
If $X$ and $Y$ are independent random variables, then $E[XY] = E[X]E[Y]$ and $Var[X+Y] = Var[X] + Var[Y]$
\end{theorem}

\begin{theorem}
	$E[Y] = \sum_{k=0}^\infty Pr[Y>k]$
\end{theorem}

\begin{theorem}[De'Morgan's Laws]
	\begin{equation*}
		\begin{split}
			Pr[\overline{A\cup B}] = Pr[\overline{A}\cap \overline{B}]\\
			Pr[\overline{A\cap B}] = Pr[\overline{A}\cup \overline{B}]
		\end{split}
	\end{equation*}
\end{theorem}



\section{Graph Theory}

\subsection*{General Graphs}
\begin{lemma}[Handshaking Lemma]
	For any graph $G$, the sum of the degrees of all vertices is twice the number of edges. That is:
	$$
	\sum_{v \in V} \deg(v) = 2|E|
	$$
\end{lemma}

\begin{theorem}
	There is an even amount of vertices with odd degree in any graph $G$.
\end{theorem}

\begin{theorem}
	Graph with $n$ vertices and $m$ edges must have $\geq n-m$ connected components
\end{theorem}

\begin{corollary}
	Each connected graph must have at least $n-1$ edges
\end{corollary}

\begin{theorem}
	Graph $G=(V,E)$ has $|V|\geq 1$ and $|E| \geq 1$, then $|V| \leq |V|^2-2|E|$
\end{theorem}

\begin{theorem}
	If $v$ is a cut vertex in $G$, $v$ is not a cut vertex in $\overline{G}$ and $\overline{G}$ is connected.
\end{theorem}

\begin{theorem}
	Either $G$ or $\overline{G}$ is connected. Where $\overline{G}$ is the complement of $G$.
\end{theorem}

\subsection*{Trees}
\begin{theorem}
	The following are equivalent for a graph $G$:
	\begin{enumerate}
		\item $G$ is a tree
		\item $G$ is connected and has $|V|-1$ edges
		\item $G$ is minimally connected, $G-e$ is disconnected for any $e\in E$
		\item $G$ is acyclic and $G+\{x,y\}$ for non-adjacent $x,y\in V$ creates a unique cycle
		\item $G$ There is a unique path between any two vertices
	\end{enumerate}
\end{theorem}

\begin{theorem}
	Every connected graph contains a spanning subtree.
\end{theorem}

\begin{theorem}
	A full binary tree with $k$ internal vertices has $2k+1$ vertices and $k+1$ leaves.
\end{theorem}

\begin{theorem}
	A binary tree of height $h$ has at most $2^h$ leaves.
\end{theorem}


\begin{theorem}
	For any graph with $\delta(G)\geq 2$, $G$ contains a cycle
\end{theorem}

\begin{theorem}
	For any graph with $\delta(G)\geq \frac{n}{2}$, $G$ is connected
\end{theorem}

\begin{theorem}
	A tree $T$ such that the degree of all vertices adjacent to a leaf is $\geq 3$ has leaf $x\neq y$ wherh $N(x) = N(y)$.
\end{theorem}

\begin{theorem}
	Every three tree with $\geq 4$ leaves has some internal vertex thats adjacent to two leaves.
\end{theorem}

\begin{theorem}
 Every three tree with $l$ has $l-2$ vertices of degree $3$.
\end{theorem}

\begin{theorem}
	A tree $T$ must have at least $\Delta(T)$ leaves.
\end{theorem}


\subsection*{Eularian \& Hamiltonian Graphs}
\begin{corollary}
	For any graph with $\delta(G)\geq \frac{n}{2}$ and $n\geq 3$, $G$ is Hamiltonian
\end{corollary}

\begin{theorem}
	If a graph contains a Eularian circuit $\rightarrow$ the line graph of $G$ is Hamiltonian
\end{theorem}

\subsection*{Planar Graphs}

\begin{theorem}[Euler's Formula]
	$n-m+f=2$ for any connected graph
\end{theorem}

\begin{theorem}
	Sum of all degrees of faces in a crossing is $2|E|$
\end{theorem}

\begin{corollary}
	$|E|\leq 3|V|-6$ for any planar graph with $|V|\geq 2$
\end{corollary}

\begin{theorem}
	All planr graph have $\delta(G)\leq 5$
\end{theorem}

\begin{theorem}
	A graph is planar $\leftrightarrow$ it contains no subdivision of $K_5$ or $K_{3,3}$
\end{theorem}

\begin{theorem}
	If a graph is planar, $\chi(G)\leq 6$
\end{theorem}

\subsection*{Graph Coloring \& Matchings}

\begin{theorem}
	A graph with $\Delta(G)\leq k$, $G$ is $k+1$-colorable
\end{theorem}

\begin{theorem}
	Graph $G$ is bipartite $\leftrightarrow$ $G$ contains no odd length cycle
\end{theorem}

\begin{theorem}
	Matching $M$ is maximum $\leftrightarrow$ contains no $M$-augmenting path
\end{theorem}

\begin{theorem}[Hall's Theorem]
	Bipartite graph $G=(X,Y,E)$ has a $X$-saturating matching $\leftrightarrow$ $|N(S)| \geq |S|$ for all $S\subseteq X$
\end{theorem}

\begin{theorem}
	A n-vertex tournament graph has $\frac{n!}{2^{n-1}}$ Hamiltonian paths
\end{theorem}

\begin{theorem}
	For graph $G=(V,E)$ and $d=\frac{2m}{n}$ is the average degree, $\alpha(G) \geq \frac{n}{2d}$
\end{theorem}

\begin{theorem}
	For graph $G$ where $|V|\geq 2$ and $\delta(G)=\delta$ contains a dominating set of at most $\frac{n\cdot (1+\ln(1+\delta))}{1+\delta}$
\end{theorem}

\begin{theorem}
	In graph $G$, For $k\in \mathbb{Z}^+$, if $n$ is larnge enough, there exists a $k$-dominated tournament on $n$ vertices.
\end{theorem}

\begin{theorem}
	For a bipartite graph $G=(X,Y,E)$, $\delta(X)\geq \Delta(Y)$ there exists a $X$-saturating matching.
\end{theorem}

\begin{theorem}
	For any graph $G=(V,E)$, there exists a partition of $V_1\cup V_2=V$ such that there are $\geq \frac{|E|}{2}$ edges between them.
\end{theorem}

\begin{theorem}
	For $G=(V,E)$, if $\delta(G)\geq 2k$, then there exists a matching of size at least $k$.
\end{theorem}

\begin{theorem}
	If every vertex in a bipartite graph $G=(X,Y,E)$ has the same degree $d\geq 0$, then $|X|=|Y|$
\end{theorem}


\begin{theorem}
	If
	$$
\left(\begin{array}{l}
n \\
k
\end{array}\right) 2^{1-\left(\begin{array}{c}
k \\
2
\end{array}\right)}<1 \text {, then } R(k, k)>n
$$
\end{theorem}

\section{Relations \& Functions}
\begin{theorem}
	$f$ is a function 
	\begin{enumerate}
		\item $f$ is injective $\leftrightarrow$ for all $a,b\in A$, $f(a)=f(b) \rightarrow a=b$
		\item $f$ is surjective $\leftrightarrow$ for all $b\in B$, there exists $a\in A$ such that $f(a)=b$
		\item $f$ is bijective $\leftrightarrow$ $f$ is injective and surjective (one-to-one correspondence)
	\end{enumerate}
\end{theorem}

\begin{theorem}
	There are $2^{n(n-1)}$ relations on a set of size $n$.
\end{theorem}

\begin{theorem}
$R$ is a relation on set $A$. $R$ is transitive $\leftrightarrow$ $R^n \subseteq R$ for all $n\in \mathbb{Z}^+$
\end{theorem}

\begin{theorem}
	$f$ and $g$ are surjective/injective/bijective $\leftrightarrow$ $f\circ g$ is surjective/injective/bijective
\end{theorem}







%-------------------------------------------------------------------------------
%	APPENDICES (optional)
%-------------------------------------------------------------------------------

%\addtocontents{toc}{\vspace{1em}} % Add a gap in the Contents, for aesthetics
\appendix
\numberedsection % Regular sections following
\input{apx/appendixA}
%\section{Appendix} \label{appB}
Est velit enim pariatur eiusmod proident commodo.
%\section{Appendix} \label{appC}
Deserunt est nostrud reprehenderit mollit cillum eu proident culpa duis Lorem eu nostrud minim.



\end{document}
